\documentclass[a4paper]{article}

\usepackage[spanish]{babel}
\usepackage[utf8]{inputenc}
\usepackage{amsmath}
\usepackage{graphicx}
\usepackage[colorinlistoftodos]{todonotes}
\usepackage{vmargin}
\usepackage{listings}

\setpapersize{A4}

\setmargins{2.5cm} % margen izquierdo
{1.5cm} % margen superior
{16.5cm} % anchura del texto
{23.42cm} % altura del texto
{10pt} % altura de los encabezados
{1cm} % espacio entre el texto y los encabezados
{0pt} % altura del pie de página
{2cm} % espacio entre el texto y el pie de página


\title{TP6-1 \\ Inteligencia Artificial \\ \large Universidad de Zaragoza}


\author{Marcos Ruiz García, 648045}

\date{\today}

\begin{document}
\maketitle

% \tableofcontents

\section{Introducción}
Se nos ha encargado realizar una serie de tareas para que el algoritmo de búsqueda Hill-Climbing permita realizar hasta 100 pasos cuando está en un llano para mejorar el porcentaje de éxito de manera notable.
\section{Tareas}
\subsection{Tarea 1}
\subsection{Tarea 2}
\subsection{Tarea 3}
\subsection{Tarea 4}

\section{Conclusión}


\end{document}